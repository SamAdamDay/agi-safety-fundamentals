%!TEX root = summaries.tex

%%%%%%%%%%%%%%%%%%%%%%%%%%%%%%%%%%%%%%%%
%% Specific Shorthands
%%%%%%%%%%%%%%%%%%%%%%%%%%%%%%%%%%%%%%%%

% Vector spaces
\newcommand{\normF}[1]{\norm{#1}_{\mathrm F}}
\DeclareMathOperator{\trace}{\mathrm{trace}}

% Probability
\DeclareMathOperator{\Cov}{\mathrm{Cov}}

% General Mathematics
\newcommand{\id}{\ensuremath{\mathrm{id}}}
\DeclareMathOperator{\dom}{\mathrm{dom}}
\DeclareMathOperator{\cod}{\mathrm{cod}}

% Common Names
\DeclareMathOperator{\Rank}{\mathrm{Rank}}
\DeclareMathOperator*{\argmax}{\mathrm{argmax}}
\DeclareMathOperator*{\argmin}{\mathrm{argmin}}
\newcommand{\new}{\mathrm{new}}

% Machine learning
\DeclareMathOperator{\softmax}{\mathrm{softmax}}
\DeclareMathOperator{\sigmoid}{\mathrm{sigmoid}}

% Computing
\DeclareMathOperator{\bigO}{\mathrm{O}}

% Statistics
\DeclareMathOperator{\cov}{\mathrm{cov}}
\DeclareMathOperator{\var}{\mathrm{var}}
\DeclareMathOperator{\covh}{\wh{\mathrm{cov}}}
\DeclareMathOperator{\varh}{\wh{\mathrm{var}}}

% Calculus
\newcommand{\der}[2]{\frac{\mathrm{d}#1}{\mathrm{d}#2}}
\newcommand{\pder}[2]{\frac{\partial #1}{\partial #2}}

% Geometry
\newcommand{\R}{\ensuremath{\mathbb{R}}}
\newcommand{\Q}{\ensuremath{\mathbb{Q}}}

% Number theory
\newcommand{\Z}{\ensuremath{\mathbb{Z}}}
\DeclareMathOperator{\lcm}{\mathrm{lcm}}
\DeclareMathOperator{\Den}{\mathrm{Den}}

% Single letters
\newcommand{\Lo}{\ensuremath{\mathcal{L}}}
\newcommand{\C}{\ensuremath{\mathbf{C}}}
\newcommand{\D}{\ensuremath{\mathcal{D}}}
\newcommand{\W}{\ensuremath{\mathcal{W}}}
\newcommand{\M}{\ensuremath{\mathcal{M}}}
\renewcommand{\P}{\ensuremath{\mathcal{P}}}

% General
\newcommand{\N}{\mathbb N}
\newcommand{\NN}{\mathbb N}
\newcommand{\NNp}{\mathbb N^{>0}}
\newcommand{\inv}[1]{{#1}^{-1}}
\newcommand{\invfs}[1]{\inv f\{#1\}}

\newcommand{\T}{\ensuremath{\mathsf{T}}}
\newcommand{\F}{\ensuremath{\mathfrak{F}}}
% \renewcommand{\S}{\ensuremath{\mathrm{S}}}
\newcommand{\E}{\ensuremath{\mathbb{E}}}
\newcommand{\A}{\ensuremath{\mathcal{A}}}
\newcommand{\Aut}{\ensuremath{\mathrm{Aut}}}
\newcommand{\vDN}{\mathrel{{\vDash}_{\!\!\!\N}}}
\newcommand{\nvDN}{\mathrel{{\nvDash}_{\!\!\!\N}}}

% Upset and downsets, with starred versions for \left and \right
% Need to be \protect'd inside \caption's
\makeatletter
\newcommand{\@usstar}[1]{{\ua}\left(#1\right)}
\newcommand{\@usnostar}[1]{{\ua}(#1)}
\newcommand{\us}{\@ifstar{\@usstar}{\@usnostar}}
\newcommand{\@dsstar}[1]{{\da}\left(#1\right)}
\newcommand{\@dsnostar}[1]{{\da}(#1)}
\newcommand{\ds}{\@ifstar{\@dsstar}{\@dsnostar}}
\newcommand{\@udsstar}[1]{{\uda}\left(#1\right)}
\newcommand{\@udsnostar}[1]{{\uda}(#1)}
\newcommand{\uds}{\@ifstar{\@udsstar}{\@udsnostar}}
\newcommand{\@Usstar}[1]{{\Ua}\left(#1\right)}
\newcommand{\@Usnostar}[1]{{\Ua}(#1)}
\newcommand{\Us}{\@ifstar{\@Usstar}{\@Usnostar}}
\newcommand{\@Dsstar}[1]{{\Da}\left(#1\right)}
\newcommand{\@Dsnostar}[1]{{\Da}(#1)}
\newcommand{\Ds}{\@ifstar{\@Dsstar}{\@Dsnostar}}
\newcommand{\@Udsstar}[1]{{\Uda}\left(#1\right)}
\newcommand{\@Udsnostar}[1]{{\Uda}(#1)}
\newcommand{\Uds}{\@ifstar{\@Udsstar}{\@Udsnostar}}
\makeatother


%%%%%%%%%%%%%%%%%%%%%%%%%%%%%%%%%%%%%%%%
%% General Shorthands
%%%%%%%%%%%%%%%%%%%%%%%%%%%%%%%%%%%%%%%%

% Single symbol shortening
\newcommand{\ep}{\epsilon}
\newcommand{\lam}{\lambda}
\newcommand{\Lam}{\Lambda}
\newcommand{\Sig}{\Sigma}
\newcommand{\sig}{\sigma}
\newcommand{\vd}{\vdash}
\newcommand{\vD}{\vDash}
\newcommand{\Vd}{\Vdash}
\newcommand{\VD}{\VDash}
\newcommand{\nvd}{\nvdash}
\newcommand{\nvD}{\nvDash}
\newcommand{\nVd}{\nVdash}
\newcommand{\nVD}{\nVDash}
\newcommand{\Dv}{\Dashv}
\newcommand{\nDv}{\nDashv}
\newcommand{\bx}{\square}
\newcommand{\la}{\leftarrow}
\newcommand{\ra}{\rightarrow}
\newcommand{\lra}{\leftrightarrow}
\newcommand{\La}{\Leftarrow}
\newcommand{\Ra}{\Rightarrow}
\newcommand{\Lra}{\Leftrightarrow}
\newcommand{\lsa}{\leftsquigarrow}
\newcommand{\rsa}{\rightsquigarrow}
\newcommand{\cra}{\circrightarrow}
\newcommand{\dra}{\dashrightarrow}
\newcommand{\dla}{\dashleftarrow}
\newcommand{\ua}{\uparrow}
\newcommand{\da}{\downarrow}
\newcommand{\uda}{\updownarrow}
\newcommand{\Ua}{\Uparrow}
\newcommand{\Da}{\Downarrow}
\newcommand{\Uda}{\Updownarrow}
\newcommand{\es}{\varnothing}
\newcommand{\sse}{\subseteq}
\newcommand{\ldot}{\ldotp}

% Paramtered command shortening
\newcommand{\ob}{\overbar}
\newcommand{\ol}{\overline}
\newcommand{\wt}{\widetilde}
\newcommand{\wh}{\widehat}
\newcommand{\bb}{\mathbb}
\newcommand{\mc}{\mathcal}
\newcommand{\mr}{\mathrm}
\newcommand{\mb}{\mathbf}

% Compound command shortening
\newcommand{\defeq}{\vcentcolon=}
\newcommand{\half}{\ensuremath{\nicefrac{1}{2}}}
\newcommand{\overbar}[1]{\mkern 1.8mu\overline{\mkern-1.8mu#1\mkern-1.8mu}\mkern 1.8mu}

% Text
\newcommand{\tail}{\ensuremath{\mathsf{tail}}}
\newcommand{\head}{\ensuremath{\mathsf{head}}}
\newcommand{\last}{\ensuremath{\mathsf{last}}}

% Math operators and paired delimiters
\DeclareMathOperator*{\btd}{\nabla}
\DeclareMathOperator*{\bdia}{\diaraw}
\DeclareMathOperator*{\bbx}{\bx}
\DeclarePairedDelimiter{\ceil}{\lceil}{\rceil}
\DeclarePairedDelimiter{\floor}{\lfloor}{\rfloor}
\DeclarePairedDelimiter{\ab}{\langle}{\rangle}
\DeclarePairedDelimiter{\sqb}{[}{]}
\DeclarePairedDelimiter{\dsb}{\llbracket}{\rrbracket}
\DeclarePairedDelimiter{\dab}{\llangle}{\rrangle}
\DeclarePairedDelimiter{\abs}{|}{|}
\DeclarePairedDelimiter{\norm}{\lVert}{\rVert}
% \newcommand{\abs}[1]{\mathopen|#1\mathclose|}

% New symbols
\newcommand{\contradiction}{\noindent
	\begin{tikzpicture}[x=0.4ex,y=0.4ex]
		\draw[line width=.15ex] (0,0) -- (1,2) -- (0,2) -- (1,4)
		(0.95,0.32) -- (0,0) -- (-0.32,0.95);
	\end{tikzpicture}\hspace*{0.2em}
}
\newcommand{\circrightarrow}{\mathrel{{\circ}\mkern-3mu{\rightarrow}}}
\newcommand{\bis}{\mathrel{\,\raisebox{.3ex}{$\underline{\makebox[.9em]{$\leftrightarrow$}}$}\,}}


%%%%%%%%%%%%%%%%%%%%%%%%%%%%%%%%%%%%%%%%
%% General Renewcommands
%%%%%%%%%%%%%%%%%%%%%%%%%%%%%%%%%%%%%%%%

\renewcommand{\leq}{\leqslant}
\renewcommand{\geq}{\geqslant}
\renewcommand{\succeq}{\succcurlyeq}
\renewcommand{\preceq}{\preccurlyeq}
\renewcommand{\nsucceq}{\not\succcurlyeq}
\renewcommand{\npreceq}{\not\preccurlyeq}
\renewcommand{\propto}{\varpropto}



% https://tex.stackexchange.com/a/337097
\newenvironment{myequation}% fake "displaystyle" environment
	{\newline\vspace{\abovedisplayskip}\hbox to \textwidth\bgroup\hss$\displaystyle}
	{$\hss\egroup\vspace{\belowdisplayskip}}

\newenvironment{inblock}[1]
	{{\setlength{\parskip}{\parsep} \par\noindent #1}
	\begin{adjustwidth}{2em}{}}
	{\end{adjustwidth}}

% Small poset diagrams
\newcommand{\FOneFork}{
	\raisebox{-0.8ex}{\resizebox{!}{2.5ex}{
		\begin{tikzpicture}
			\node[world] (z) at (0,0) {};
			\node[world] (a) at (0,1) {};
			\path[draw] (z) -- (a);
		\end{tikzpicture}
	}}
}
\newcommand{\FPoint}{
	\raisebox{0.2ex}{\resizebox{!}{0.9ex}{
		\begin{tikzpicture}
			\node[world] (z) at (0,0) {};
		\end{tikzpicture}
	}}
}

\newcommand{\FTwoFork}{
	\raisebox{-0.8ex}{\resizebox{!}{2.5ex}{
		\begin{tikzpicture}
			\node[world] (z) at (0,0) {};
			\node[world] (a1) at (-0.7,1) {};
			\node[world] (b1) at (0.7,1) {};
			\path[draw] (z) -- (a1);
			\path[draw] (z) -- (b1);
		\end{tikzpicture}
	}}
}

\newcommand{\FThreeFork}{
	\raisebox{-0.5ex}{\resizebox{!}{2ex}{
		\begin{tikzpicture}
			\node[world] (z) at (0,0) {};
			\node[world] (a1) at (-1,1) {};
			\node[world] (b1) at (0,1) {};
			\node[world] (c1) at (1,1) {};
			\path[draw] (z) -- (a1);
			\path[draw] (z) -- (b1);
			\path[draw] (z) -- (c1);
		\end{tikzpicture}
	}}
}

\newcommand{\FScott}{
	\raisebox{-1.1ex}{\resizebox{!}{3ex}{
		\begin{tikzpicture}
			\node[world] (z) at (0,0) {};
			\node[world] (a1) at (-0.7,1) {};
			\node[world] (a2) at (-0.7,2) {};
			\node[world] (b1) at (0.7,1) {};
			\path[draw] (z) -- (a1) -- (a2);
			\path[draw] (z) -- (b1);
		\end{tikzpicture}
	}}
}



\DeclareFontFamily{U} {MnSymbolC}{}

\DeclareFontShape{U}{MnSymbolC}{m}{n}{
	<-6> MnSymbolC5
	<6-7> MnSymbolC6
	<7-8> MnSymbolC7
	<8-9> MnSymbolC8
	<9-10> MnSymbolC9
	<10-12> MnSymbolC10
	<12-> MnSymbolC12}{}
\DeclareFontShape{U}{MnSymbolC}{b}{n}{
	<-6> MnSymbolC-Bold5
	<6-7> MnSymbolC-Bold6
	<7-8> MnSymbolC-Bold7
	<8-9> MnSymbolC-Bold8
	<9-10> MnSymbolC-Bold9
	<10-12> MnSymbolC-Bold10
	<12-> MnSymbolC-Bold12}{}

\DeclareSymbolFont{MnSyC} {U} {MnSymbolC}{m}{n}
\DeclareMathSymbol{\meddiamond}{\mathbin}{MnSyC}{110}

\newcommand{\raisedia}[2]{\raisebox{0.8\depth}{$#1\meddiamond$}}
\newcommand{\diaraw}{\mathpalette\raisedia\relax}
\newcommand{\dia}{{\diaraw}}
\newcommand{\sdia}{{\scalebox{0.7}{$\dia$}}}