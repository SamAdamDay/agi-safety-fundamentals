%!TEX root = ../summaries.tex

\chapter{Decomposing tasks for outer alignment}

\section{Factored cognition}

\url{https://ought.org/research/factored-cognition}


\subsection{Introduction}

\begin{itemize}
    \item \emph{One-step amplification}: agent has a number of copies of itself which work on sub-tasks with equal limitations on computational capacity.
    \item \emph{Iterated amplification}: this iterated.
    \item \emph{Factored cognition}: learning broken down like this into small and mostly independent tasks.
\end{itemize}


\subsection{Scalable mechanisms for solving cognitive tasks}

\begin{itemize}
    \item Want to find mechanisms for solving cognitive tasks which scale with respect to number of human work-hours and access to ML algorithms.
    \item Assumptions.
    \begin{enumerate}[label=(\arabic*)]
        \item\label{item:well-motivated; assumptions; factored cognition} Human workers well-motivated.
        \item\label{item:limited time; assumptions; factored cognition} Each only available for say 15 minutes.
        \item\label{item:same background; assumptions; factored cognition} Each has same background knowledge.
    \end{enumerate}
    \item Mechanism: \emph{Iterated Distillation-Amplification}.
    \item Example cognitive tasks.
    \begin{itemize}
        \item Read this book and tell me why $x$ did $y$.
        \item Provide a detailed analysis of the pros and cons of these two products.
        \item Tell me how to invest \$100k to achieve the most social good.
        \item Write a paper on NLP that substantially advances the state of the art.
    \end{itemize}
    \item Scalability.
    \begin{itemize}
        \item More resources $\rsa$ better results.
        \item Resources: human work-hours and ML algos.
        \item Better: more aligned with task-setter's interests.
        \item Scalability desirable because it helps turn thinking into a commodity.
        \item Scalable system automatically gets more helpful when we plug in more advanced ML systems.
    \end{itemize}
    \item Organizing human work on cognitive tasks.
    \begin{itemize}
        \item Scalability of single person doing task limited by number of hours available and ability to reason and learn.
        \item Scalability of group of people limited by communication and delegation.
        \item Doesn't discuss motivation problem (assumption \ref{item:well-motivated; assumptions; factored cognition}).
        \item How do we orchestrate people so that the output scales with number of people.
        \item Short-term context-free work.
        \begin{itemize}
            \item By assumption \ref{item:limited time; assumptions; factored cognition}, no worker has time to build up much context.
            \item Can we compose simple local tasks to solve a complex problem?
        \end{itemize}
        \item Coordination of short-term work as algorithm design.
        \begin{itemize}
            \item Take humans to be a function which takes a task string and outputs result (assumption \ref{item:same background; assumptions; factored cognition}).
            \item Can we mechanically compose calls to this stateless function to solve task?
            \item This is about algorithm design.
        \end{itemize}
        \item Matching the quality of any other approach to solving cognitive tasks.
        \begin{itemize}
            \item Want approach that scales with number of calls to function.
            \item Could compare to other ways of solving task.
            \begin{itemize}
                \item Is there a number of calls to function with means we can do as well as any other fixed solution?
                \item Problem: evaluating how well a task is solved a hard cognitive question.
                \item Problem: empirical content of solutions, which may include built-in solutions to some tasks.
            \end{itemize}
            \item Alternative: compare to subjective of idealised deliberation.
            \begin{itemize}
                \item Problem: this is exactly what we're trying to solve.
            \end{itemize}
            \item Best compromise: can we solve any task to arbitrary high quality?
            \begin{itemize}
                \item Unlikely to be the case: any task which involves learning could be done without learning.
                \item Might still be useful to aim for it, to get systems which scale well in practice, but have theoretical bounds.
                \item Also: might produce alternatives to solutions which are expensive to implement directly.
            \end{itemize}
        \end{itemize}
    \end{itemize}
    \item Applying machine learning to cognitive tasks.
    \begin{itemize}
        \item Scalability now also wrt to ML sophistication.
        \begin{itemize}
            \item Better priors.
            \item Better inference.
            \item Better training paradigms.
        \end{itemize}
        \item Approaches that don't scale.
        \begin{itemize}
            \item Training systems on $(\text{task}, \text{solution})$ pairs.
            \begin{itemize}
                \item Doesn't scale because we can't generate an arbitrary quantity of training data: can't generate solutions of arbitrarily high quality.
            \end{itemize}
            \item Reinforcement learning based on how good a solutions seems.
            \begin{itemize}
                \item Optimises for proxy to solution's goodness.
            \end{itemize}
        \end{itemize}
        \item An approach that might scale.
        \begin{itemize}
            \item Worthwhile to consider how to scalably apply current ML algos, assuming they only scale along the dimensions mentioned above.
            \item Iterated Distillation-Amplification.
            \begin{enumerate}[label=\arabic*.]
                \item Initialise fast ML $A$ randomly.
                \item Repeat:
                \begin{enumerate}[label=\alph*.]
                    \item Amplification: Build slow system which involves human making single step, with multiple calls to $A$.
                    \item Distillation: Retrain $A$ to imitate behaviour of this slow system.
                \end{enumerate}
            \end{enumerate}
            \begin{itemize}
                \item Depends on ability to decompose task into small context-free steps.
            \end{itemize}
        \end{itemize}
    \end{itemize}
\end{itemize}